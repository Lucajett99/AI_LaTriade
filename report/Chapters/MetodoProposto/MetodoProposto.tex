\documentclass[../Report.tex]{subfiles}
\usepackage[italian]{babel}
\begin{document}
    \chapter{Metodo Proposto}
    Questo studio ha come obiettivo quello di andare a valutare la possibilità di predire l'abbandono universitario di uno studente prima che questo avvenga. Per farlo siamo andati a valutare tre possibili momenti di previsione, il momento dell'iscrizione, la fine del primo semestre del primo anno di corso e la fine del primo anno di corso. La scelta sui tre modelli da utilizzare, cioè Regressione Logistica, Random Forest e Support Vector Machine, è stata dettata da interessi personali nello studio del funzionamento di questi modelli e dal fatto che questi siano i modelli più utilizzati in letteratura per risolvere problemi simili di predizione. Inoltre, un'altro importante motivo alla base di questa scelta è il fatto che siano stati utilizzati dall'altro lavoro che utilizza il Dataset dell'UniBo \cite{DelBonifro} per avere un riferimento per il confronto dei risultati ottenuti.\\
    Successivamente abbiamo fatto un tentativo di Ensamble Learning utilizzando lo Stacking Algorithm con l'idea di migliorare le prestazioni rispetto ai tre modelli.
    
    \subfile{../LogisticRegression/logisticRegression.tex}
    \subfile{../SVM/SVM.tex}
    \subfile{../RandomForest/RandomForest.tex}
    \subfile{../StackingAlgorithm/StackingAlgorithm.tex}
    
\end{document}